\chapter[DNA sequencing]{DNA sequencing and MinION technology}
\label{kap:bio}

In this chapter we introduce DNA sequencing and problems of new MinION technology.

\section{DNA Sequencing}

Information inside DNA is composed of four nitrogenous base adenine, guanine, cytosine, and thymine. We represent them with single letters A, C, T, G.

DNA sequencing is the process of determining the order of these nucleotides within a DNA molecule. 
...
... 


\section{MinION technology}

MinION is a new real-time sequencing technology developed by Oxford Nanopore
Technologies. It is based on nanopores and its highly portable. 

A nanopore is a hole so small, that only single DNA strand fits in. MinION
uses a protein nanopore set in an electrically resistant polymer membrane. An ionic current is passed through the nanopore by setting a voltage across this membrane. If an analyte passes through the pore or near its aperture, this event creates a characteristic disruption in current. Measurement of that current makes it possible to identify the molecule in question.



Process of translating signal into a sequence of bases (letters A, C, G, T) is called base calling. It is slow process and introduces a lot of errors.