\chapter{Sequence Alignment}

\label{kap:alignment} % id kapitoly pre prikaz ref

Alignment is the task of locating equivalent regions of two or more sequences to find their similarity.
The large similarity often mean the same biological function or common ancestor in the evolutionary history.
In this chapter, we formalize the problem of alignment and present standard algorithms for
solving this problem.

\section{Pairwise alignment}

\theoremstyle{definition}
\newtheorem{definition}{Definition}[section]
\theoremstyle{definition}
\begin{definition}[Alignment]
Let $u=u_1 \dots u_n$, $v=v_1 \dots v_m$ be two sequences where $u_i,v_i\in \{A, C, T, G\}$ and $M$ be a matrix
$$M=
\begin{pmatrix}
  M_{1,1} & M_{1,2} & \dots & M_{1,k} \\
  M_{2,1} & M_{2,2} & \dots & M_{2,k}
\end{pmatrix}$$
We call $M$ an alignment of $u$ and $v$ if:
\begin{enumerate}
\item $\forall i,j : M_{i,j}\in \{A,C,T,G,-\}$
\item $M_{1,1} M_{1,2} \dots M_{1,k}$ is a word created by inserting dashes into $u$
\item $M_{2,1} M_{2,2} \dots M_{2,k}$ is a word created by inserting dashes into $v$
\item No column contains two dashes
\end{enumerate}
\end{definition}

For example 
\setlength{\arraycolsep}{1pt}
\setcounter{MaxMatrixCols}{20}
$$
\begin{pmatrix}
   - & G & T & A & C & G & T & C & C & T & A & A \\
   T & G & T & A & C & G & C & C & - & T & - & -
\end{pmatrix}$$
is one  of many possible alignments of sequences $GTACGTCCTAA$ and $TGTACGCCT$.
Another alignment of these sequences could be 
$$\begin{pmatrix}
   -&-&-&-&G&T&A&C&G&T&C&C&T&A&A \\
   T&G&T&A&C&G&C&C&-&-&-&-&T&-&-
\end{pmatrix}$$
To find the best alignment (with most of the similarities), we define a scoring system which assigns a numeric value to the alignment.

A basic scoring system can be $+1$ for each matching column, $-1$ for every other column.

\begin{definition}[Global alignment]
Given two sequences and a scoring system, the global alignment is the alignment with the highest score in the scoring system.
\end{definition}

Sometimes, it is better to search for the most similar part of the two sequences. We call this approach the local alignment

\begin{definition}[Local alignment]
Given two sequences $u$, $v$ and a scoring system. Let $u'$ and $v'$ be substrings of $u$ and $v$ such that their global alignment has the highest score among all substrings of $u$ and $v$. Local alignment of $u$ and $v$ is the global alignment of $u'$ and $v'$.
\end{definition}

\section{Needleman-Wunsch algorithm}
\label{nwalg}
Needleman-Wunsch algorithm \cite{needleman} for computing global alignment is based on dynamic programming. 

Let $u=u_1 \dots u_n$, $v=v_1 \dots v_n$ be two sequences. We construct matrix $A[n,m]$ where $A[i,j]$ is the global alignment score of $u_1 \dots u_i$ and $v_1 \dots v_j$.

Let $s(i,j)$ be score of alignment of $u_i$ and $v_j$ ($1$ if they are equal, $-1$ otherwise).
Matrix $A$ is constructed dynamically row by row, where $A[i,j]$ is computed as the maximum of three possibilities:
\begin{enumerate}
\item $A[i-1,j-1]+s(i,j)$
\item $A[i-1,j]-1$
\item $A[i,j-1]-1$
\end{enumerate}

In the first case, we align $u_i$ to $v_j$ as a match or mismatch. In the second case, we insert dash into the first sequence. In the third case, we insert dash into the second sequence.

The final alignment score is $A[n,m]$ and the alignment can be reconstructed by finding the path of computation from $A[n,m]$ to $A[1,1]$.
This path can be simply found by following the highest values out of the three possibilities used to compute the values in matrix $A$. 

Needleman-Wunsch algorithm can be easily modified to find the local alignment in two steps.
First, we add the fourth case:
\begin{enumerate}
 \setcounter{enumi}{3}
\item $0$
\end{enumerate}
This allows us to change the starting point of the alignment, if doing so leads to a better solution.
Second, the final score is the maximum value in the matrix, which allows us to change ending points of the alignment.
Reconstruction of the alignment is done by following the path from the maximum value in the matrix to the closest $0$, 
where the local alignment begins.

\section{Multiple alignment}
If we have a large dataset of evolutionarily related DNA sequences that share a common ancestor, we might find something about such ancestor, if we look at their overall similarity.

\begin{definition}[Multiple alignment]
Let $u_1=u_{1,1} \dots u_{1,m_1}, \ldots, u_n=u_{n,1} \dots u_{n,m_n}$ be $n$ sequences where $u_{i,j} \in \{A, C, T, G\}$ and $M$ be a matrix
$$M=
\begin{pmatrix}
  M_{1,1} & M_{1,2} & \dots & M_{1,k} \\
  M_{2,1} & M_{2,2} & \dots & M_{2,k} \\
  \vdots&&&\vdots\\
  M_{n,1} & M_{n,2} & \dots & M_{n,k} \\
\end{pmatrix}$$
We call $M$ an alignment of $u_1, \dots, u_n$ if:
\begin{enumerate}
\item $\forall i,j : M_{i,j}\in \{A,C,T,G,-\}$
\item $M_{i,1} M_{i,2} \dots M_{i,k}$ is a word created by inserting dashes into $u_i$
\item No column contains $n$ dashes
\end{enumerate}
\end{definition}

We can see that the alignment of two sequences is a special case of a multiple alignment.

\subsection{Dynamic programming}
We can modify Needleman-Wunsch algorithm to solve this problem.
Matrix $A$ would become $n$-dimensional and the time complexity of computing such matrix would be $O(\prod\limits_{i=1}^{n} m_{i})$.
Matrix $A$ can still be constructed dynamically row by row, where $A[i_1,...i_n]$ is computed as the maximum of $n+1$ possibilities:
\begin{enumerate}
\item $A[i_1-1,i_2-1, \dots, i_n-1]+s(i,j)$
\item $A[i_1-1, i_2, i_3, \dots, i_n]-1$
%\item $A[i_1, i_2-1, i_3, \dots, i_n]-1$
%\end{enumerate}
\renewcommand{\labelenumi}{\alph{enumi}+1.}
%\begin{enumerate}
\setcounter{enumi}{10}
\item $A[i_1,\dots, i_k-1, \dots, i_n]-1$
\setcounter{enumi}{13}
\item $A[i_1, i_2, i_3, \dots, i_n-1]-1$
\end{enumerate}

To find the global optimum for $n$ sequences has been shown to be an NP-complete problem \cite{wang1994complexity}.

\subsection{Heuristics}
Since DNA sequences are usually large, exponential time complexity is unacceptable.
Several heuristic algorithms were developed, with different time complexities and accuracy.

\subsubsection{Progressive alignment}
The most common heuristic approach to multiple alignment is a progressive technique \cite{progressive}.
Progressive alignment builds up the final alignment by combining pairwise alignments.
A binary tree called a guide tree with sequences as leafs decides an order of pairwise alignments.
The final alignment is built by following that tree from leafs to root.
Such tree can be composed by finding most similar pairs among all possible pairwise alignments resulting into
a quadratic number of alignments.
The problematic part of this approach is aligning two alignments which can be done in many various ways \cite{baum}.

\subsubsection{Iterative methods}
Another method is to construct the alignment by adding sequences one by one to the final alignment.
This way, we only need a linear number of alignments, but we are aligning one sequence to an alignment, which can again be done in various ways.
Efficiency is improved at the cost of accuracy \cite{iterative}.