\chapter{Sequence Alignment}

\label{kap:alignment} % id kapitoly pre prikaz ref

Alignment is the task of locating equivalent regions of two or more sequences to find their similarity.
The large similarity often mean the same biological function or common ancestor in evolution history.
In this chapter we formalize problem of alignment and present standard algorithms
solving this problem.

\section{Pairwise alignment}

\theoremstyle{definition}
\newtheorem{definition}{Definition}[section]
\theoremstyle{definition}
\begin{definition}{Alignment}
Let $u=u_1 \dots u_n$, $v=v_1 \dots v_m$ be two sequences where $u_i,v_i\in \{A, C, T, G\}$ and $M$ be a matrix
$$M=
\begin{pmatrix}
  M_{1,1} & M_{1,2} & \dots & M_{1,k} \\
  M_{2,1} & M_{2,2} & \dots & M_{2,k}
\end{pmatrix}$$
We call $M$ an alignment of $u$ and $v$ if:
\begin{enumerate}
\item $\forall i,j : M_{i,j}\in \{A,C,T,G,-\}$
\item $M_{1,1} M_{1,2} \dots M_{1,k}$ is a word created by inserting dashes into $u$
\item $M_{2,1} M_{2,2} \dots M_{2,k}$ is a word created by inserting dashes into $v$
\item No column contain two dashes
\end{enumerate}
\end{definition}

For example 
\setlength{\arraycolsep}{1pt}
\setcounter{MaxMatrixCols}{20}
$$
\begin{pmatrix}
   - & G & T & A & C & G & T & C & C & T & A & A \\
   T & G & T & A & C & G & C & C & - & T & - & -
\end{pmatrix}$$
is one  of many possible alignments of sequences $GTACGTCCTAA$ and $TGTACGCCT$.
Another alignment of these sequences could be 
$$\begin{pmatrix}
   -&-&-&-&G&T&A&C&G&T&C&C&T&A&A \\
   T&G&T&A&C&G&C&C&-&-&-&-&T&-&-
\end{pmatrix}$$
To find the best alignment (with most of similarities), we define a scoring system which assigns a numeric value to alignment.

Basic scoring system can be $+1$ for each matching column, $-1$ for every other column.

\begin{definition}{Global alignment}
Given two sequences and scoring system, global alignment is alignment with highest score in scoring system.
\end{definition}

Sometimes, it is better to search for most similar part of two sequences. We call this approach Local alignment

\begin{definition}{Local alignment}
Given two sequences $u$, $v$ and scoring system. Let $u'$ and $v'$ be substrings of $u$ and $v$ such, that their global alignment has highest score among all substrings of $u$ and $v$. Local alignment of $u$ and $v$ is global alignment of $u'$ and $v'$.
\end{definition}

\subsection{Algorithm}
Typically used algorithm for computing global alignment is Needleman-Wunsch algorithm.

It is based on dynamic programming. 

Let $u=u_1 \dots u_n$, $v=v_1 \dots v_n$ be two sequences. We construct matrix $A[n,m]$ where $A[i,j]$ is global alignment score of $u_1 \dots u_i$ and $v_1 \dots v_j$.

Let $s(i,j)$ be score of alignment of $u_i$ and $v_j$ ($1$ if they are equal, $-1$ otherwise).
Matrix $A$ is constructed dynamically row by row, where $A[i,j]$ is computed as maximum of three possibilities:
\begin{enumerate}
\item $A[i-1,j-1]+s(i,j)$
\item $A[i-1,j]-1$
\item $A[i,j-1]-1$
\end{enumerate}

In first possibility we align $u_i$ to $v_j$ as a match or mismatch, in second we insert dash into first sequence and in third we insert dash into second sequence.

Final alignment score is $A[n,m]$ and alignment can be reconstructed by finding path of computation from $A[n,m]$ to $A[1,1]$.

This path can be simply found by following highest values of three possibilities used to compute values in matrix $A$. 


Needleman-Wunsch algorithm can be simple modified to find local alignment by two steps.
Firstly, we add fourth possibility:
\begin{enumerate}
 \setcounter{enumi}{3}
\item $0$
\end{enumerate}
This allows to change starting point of alignment, if that is better.
Secondly, final score is maximum value in matrix. That allows to change ending point of alignment.

Reconstruction of alignment is done by finding path from this maximum value to $0$ somewhere in matrix $A$.

\section{Multiple alignment}
Once we have large dataset of evolutionary related DNA sequences that share common ancestor, we might find something about such ancestor, if we look on their overall similarity.

\begin{definition}{Multiple alignment}
Let $u_1=u_{1,1} \dots u_{1,m_1}, \ldots, u_n=u_{n,1} \dots u_{n,m_n}$ be $n$ sequences where $u_{i,j} \in \{A, C, T, G\}$ and $M$ be a matrix
$$M=
\begin{pmatrix}
  M_{1,1} & M_{1,2} & \dots & M_{1,k} \\
  M_{2,1} & M_{2,2} & \dots & M_{2,k} \\
  \vdots&&&\vdots\\
  M_{n,1} & M_{n,2} & \dots & M_{n,k} \\
\end{pmatrix}$$
We call $M$ an alignment of $u_1, \dots, u_n$ if:
\begin{enumerate}
\item $\forall i,j : M_{i,j}\in \{A,C,T,G,-\}$
\item $M_{i,1} M_{i,2} \dots M_{i,k}$ is a word created by inserting dashes into $u_i$
\item No column contain $n$ dashes
\end{enumerate}
\end{definition}

We can see that alignment of two sequences is a special case of multiple alignment.

\subsection{Dynamic programming}
We can modify Needleman-Wunsch algorithm to solve this problem.
Matrix $A$ would become $n$-dimensional and complexity of computing such matrix would be $O(max(m_i)^n)$.

To find the global optimum for $n$ sequences has been shown to be an NP-complete problem.\cite{wang1994complexity}

\subsection{Heuristics}
As DNA data are usually big, exponencial complexity is unacceptable.
Several heuristic algorithms were developed with different time complexity and accuracy.

\subsubsection{Progressive alignment}
Most common heuristic approach to multiple alignment is progressive technique.
Progressive alignment builds up a final alignment by combining pairwise alignments.
A guide tree which decides order of pairwise alignments is built and final aligning follows this tree.
Such tree can by composed by finding most similar pairs among all possible pairwise alignments resulting into
quadratic number of alignments.
Problematic part of this approach is aligning two alignments which can be done in many various ways.

\subsection{Iterative methods}
Another method is to construct alignment by adding sequences one by one to final alignment.
This way we only need linear number of alignments but we are aligning sequence to alignment which can again be done in various ways.
Efficiency is improved at the cost of accuracy.
