\chapter*{Introduction} % chapter* je necislovana kapitola
\addcontentsline{toc}{chapter}{Introduction} % rucne pridanie do obsahu
\markboth{Introduction}{Introduction} % vyriesenie hlaviciek

DNA sequencing is a great challenge of modern science. 
Ten years ago, it took several months and millions of dollars to sequence whole human genome. 
Modern technologies can do so in days at cost of thousands of dollars.
These technologies are very precise, but the device is still very 
big and expensive.


In 2012, scientists at Oxford Nanopore Technologies developed the new portable device called MinION
that can sequence DNA and costs only several thousands. 
This device allows us to sequence genomes at International space station or in extreme conditions on demand. 
MinION has, however, several issues. The main problem is precision of sequencing. This device measures a current flowing through nanopore with DNA molecule inside. The process of translating this data into DNA sequence is called basecalling and introduces a 
lot of errors due to disruptions in measured values. 

Our goal is 
to get away from these disruptions by taking multiple reads of same 
DNA sequence and producing a single signal with fewer disruptions 
and higher quality. To do this we adapted approaches known from 
multiple alignment of DNA sequences along with dynamic time warping used to align signals.

In the first chapter, we introduce biological motivation and aspects 
of a MinION device along with standard data processing procedure for this data.

In the second chapter, we described known algorithms 
for sequence alignment, multiple sequence alignment, and alignment of signals.

In the third chapter, we describe our approaches to combine multiple 
sequence alignment with the alignment of signals and reconstruction of 
a signal from this alignment.

The fourth chapter shows visualizations we created to analyze the outcome of our approaches.

In the final chapter, we describe testing data and testing procedure to evaluate the quality of our approaches.